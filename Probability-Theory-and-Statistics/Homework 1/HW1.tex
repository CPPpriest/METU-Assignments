\documentclass[12pt]{article}
\usepackage[utf8]{inputenc}
\usepackage{float}
\usepackage{amsmath}

\usepackage{listings}

\usepackage[hmargin=3cm,vmargin=6.0cm]{geometry}
%\topmargin=0cm
\topmargin=-2cm
\addtolength{\textheight}{6.5cm}
\addtolength{\textwidth}{2.0cm}
%\setlength{\leftmargin}{-5cm}
\setlength{\oddsidemargin}{0.0cm}
\setlength{\evensidemargin}{0.0cm}

%misc libraries goes here

\begin{document}

\section*{Student Information } 
%Write your full name and id number between the colon and newline
%Put one empty space character after colon and before newline
Full Name :  Ömer Kılınç\\
Id Number :  2448603\\

% Write your answers below the section tags
\section*{Answer 1}

\subsection*{a)} 

\[ \sum_{n=1}^{5} \frac{N}{x} = 1 \]

\[ N*(\frac{1}{1} + \frac{1}{2} + \frac{1}{3} + \frac{1}{4} + \frac{1}{5}) = 1 \]

\[ N*(\frac{60}{137} ) = 1 \]

\[ N = 0.438 \]


\subsection*{b)} 

\[ \sum_{n=1}^{5} \ xP(x) = \sum_{n=1}^{5} \ N = 5N = 2.19 \]

\subsection*{c)} 

\[ Var(X) = E(X^{2}) - \mu^{2} \]

\[ = (\sum_{n=1}^{5} \ x^{2}P(x) ) - \mu^{2} \]

\[ = (\sum_{n=1}^{5} \ xN) - \mu^{2} \]

\[ = 15N - \mu^{2} \]

\[ = 6.57 - (2.19)^{2} \]

\[ = 1.774 \]

\subsection*{d)} 

\[ E(Y) = \sum_{n=1}^{5} \ yP(y) = \sum_{n=1}^{5} \ \frac{y^{2}}{15} = \frac{1}{15} * ( 1^{2} + 2^{2} + 3^{2} + 4^{2} + 5^{2}) \]

\[ E(Y)  = 3.667 \]

\[ E(XY) = \sum_{n=1}^{5} \ xyP(x,y) = \sum_{n=1}^{5} \ xyP(x)P(y)  \]

\[ \sum_{n=1}^{5} \ N\frac{y^{2}}{15} = 8.03 \]

\[ Cov(X,Y) = E(XY) - E(X)E(Y) \]
\[ = 8.03 - (2.19)(3.667) = 0\]  
\[ \text{ Since P(x,y) = P(x)P(y) ;  X and Y are independent random variables.} \]
\[ \text{As a result, E(XY) = E(X)E(Y)  and Cov(X,Y) = E(XY) - E(X)E(Y) = 0.}\]
\[ \text{Conclusion: Covariance of two independent random variables is zero.} \]



\section*{Answer 2}


\subsection*{a)} 

\[ \text{X: Number of success in 1000 trials.} \]

\[ P(X \ge 1) = 1 - P(X < 1) = 1- F(0) = 0.95 \]
\[ F(0) = 0.05 \]
\[ \text{ For F(0) = 0.05, }  \lambda = 3 \text{ , using Poisson approximation to Binomial } \]

\[ \lambda = np = 1000p = 3 \]
\[ p = 0.003 \]

\subsection*{b)} 
\subsubsection*{(i)} 
\[ \text{Y: Number of games to play in order to win two times} \]

\[ P(Y > 500) \]
\[ \text{X: Number of wins against IM in 500 games.} \]
\[ \text{ (Negative Binomial to Binomial)} \] 
\[ P(Y > 500 ) = P(X < 2) = F(1) \] 

\[ \text{Using Poisson approximation to Binomial with } \]
\[ \lambda = np = 500 * 3 * 10^{-1} = 1.5 \]
\[ F(1) = 0.558\]

\subsubsection*{(ii)}

\[ \text{Y: Number of games to play in order to win two times} \]

\[ P(Y > 10000) \]
\[ \text{X: Number of wins against GM in 10000 games.} \]
\[ \text{ (Negative Binomial to Binomial)} \] 
\[ P(Y > 10000 ) = P(X < 2) = F(1) \] 

\[ \text{Using Poisson approximation to Binomial with } \]
\[ \lambda = np = 10000 * 10^{-4} = 1 \]
\[ F(1) = 0.736\]


\subsection*{c)} 
\[ \text{Y: Number of days that one feels sick} \]
\[ P(Y \le 6) = F(6) \text{ with } \lambda = np = (366)(0.02) = 7.32\]
\[ F(6) \text{ for } \lambda = 7.32  \cong \frac{ (0.450 + 0.378)}{2} = 0.414 \]

\section*{Answer 3}
\subsection*{a)} 

\begin{lstlisting}
#Total number of days
n = 366; 

#probability of not feeling sick on any given day
p = 0.98; 


#sum pmf from x = 360 to x = 366

totalProb = 0;

for k = 360:366
    prob_k = nchoosek(n, k) * p^k * (1-p)^(n-k);
    totalProb = totalProb + prob_k;
end

#Print result
fprintf('total = %.3f\n', totalProb);

\end{lstlisting}

\textbf{total = 0.401}

\[ \textbf{The found value is lower than the calculated one due to approximation.}\]
\[ \textbf{The poisson approximated value is slightly higher than preciously calculated Binomial value.}\]
\subsection*{b)} 

\begin{lstlisting}
p = 0.98;
x = 6;

#Range variable for n
N = 50:400;

#Array: Binomial cdf values for ranging n
cdfBinom = zeros(length(N), 1);

#Array: Poisson cdf values for ranging n
cdfPoisson = zeros(length(N), 1);

#Fill arrays for ranging n
for i = 1:length(N)
    n = N(i);
    
    cdfBinom(i) = binocdf(x, n, 0.02);
    
    cdfPoisson(i) = poisscdf(x, (n * 0.02));
end


figure;

plot(N, cdfBinom, 'b-', 'LineWidth', 2);
hold on;
plot(N, cdfPoisson, 'r--', 'LineWidth', 2);

xlabel('n');
ylabel('P');

legend('Binomial', 'Poisson');

grid on;
\end{lstlisting}

\textbf{See Figure 1}

\subsection*{c)} 
\begin{lstlisting}
p = 0.78;
x = 6;

#Range variable for n
N = 50:400;

#Array: Binomial cdf values for ranging n
cdfBinom = zeros(length(N), 1);

#Array: Poisson cdf values for ranging n
cdfPoisson = zeros(length(N), 1);

#Fill arrays for ranging n
for i = 1:length(N)
    n = N(i);
    
    cdfBinom(i) = binocdf(x, n, 0.22);
    
    cdfPoisson(i) = poisscdf(x, (n * 0.22));
end


figure;

plot(N, cdfBinom, 'b-', 'LineWidth', 2);
hold on;
plot(N, cdfPoisson, 'r--', 'LineWidth', 2);

xlabel('n');
ylabel('P');

legend('Binomial', 'Poisson');

grid on;
\end{lstlisting}

\textbf{See Figure 2}
\[\textbf{The total probability drastically decreases with p.} \]
\[\textbf{The difference between Binomial and Poisson CDF is noticable.} \]
\end{document}

