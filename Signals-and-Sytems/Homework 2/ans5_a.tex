\documentclass[12pt]{article}
\usepackage[utf8]{inputenc}

\usepackage{float}
\usepackage{amsmath}
\usepackage{fullpage}
\usepackage{amsfonts, amsmath, pifont}
\usepackage{amsthm}
\usepackage{graphicx}
\usepackage{float}

\usepackage{tkz-euclide}
\usepackage{tikz}
\usepackage{pgfplots}
\pgfplotsset{compat=1.13}


\usepackage[hmargin=3cm,vmargin=6.0cm]{geometry}
%\topmargin=0cm
\topmargin=-2cm
\addtolength{\textheight}{6.5cm}
\addtolength{\textwidth}{2.0cm}
%\setlength{\leftmargin}{-5cm}
\setlength{\oddsidemargin}{0.0cm}
\setlength{\evensidemargin}{0.0cm}

%misc libraries goes here




\begin{document}

\section*{Answer 5}

\[ y[n] = \frac{1}{5}y[n-1] + 2x[n-2] \]

\subsection*{a)} 
\[ \textbf{Feeding the system y[n] with unit impulse signal: } \]
\[ y[n] \to h[n]  \textbf{  for  }  x[n] \to \delta[n]  \]

\[ h[n] = \frac{1}{5}h[n-1] + 2 \delta [n-2]  \]

\[ \textbf{System is initially at rest: } h[0] = 0 \]
\[ \textbf{By using the recursive method, we can obtain the impulse response. }\]
\[ h[0] = 0 \]
\[ h[1] = \frac{1}{5}h[0] + 2\delta [-1] = 0 \]
\[ h[2] = \frac{1}{5}h[1] + 2\delta [0] = 2 \]
\[ h[3] = \frac{1}{5}h[2] + 2\delta [1] = (\frac{1}{5})(2) \]
\[ h[4] = \frac{1}{5}h[3] + 2\delta [2] = (\frac{1}{5})^{2}(2) \]
\[ h[5] = \frac{1}{5}h[4] + 2\delta [3] = (\frac{1}{5})^{3}(2) \]
\[ ... \]
\[ h[n] = (\frac{1}{5})^{n-2}(2) \]
\[\text{ for } n > 1 \]  

\[ h[n] = (\frac{1}{5})^{n-2}(2) \mu [n-2] \]



\end{document}


