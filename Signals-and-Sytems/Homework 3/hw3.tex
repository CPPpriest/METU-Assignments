\documentclass[10pt,a4paper, margin=1in]{article}
\usepackage{fullpage}
\usepackage{amsfonts, amsmath, pifont}
\usepackage{amsthm}
\usepackage{graphicx}
\usepackage{float}

\usepackage{tkz-euclide}
\usepackage{tikz}
\usepackage{pgfplots}
\pgfplotsset{compat=1.13}

\usepackage{geometry}
 \geometry{
 a4paper,
 total={210mm,297mm},
 left=10mm,
 right=10mm,
 top=10mm,
 bottom=10mm,
 }
 % Write both of your names here. Fill exxxxxxx with your ceng mail address.
 \author{
  Kılınç, Ömer\\
  \texttt{e2448603@ceng.metu.edu.tr}
}

\title{CENG 384 - Signals and Systems for Computer Engineers \\
Spring 2024 \\
Homework 3}
\begin{document}
\maketitle



\noindent\rule{19cm}{1.2pt}

\begin{enumerate}

\item %write the solution of q1
    \[ a_k = \begin{cases} 
        -1 & \textbf{k is even} \\
        1 & \textbf{k is odd}
        \end{cases}
    \]

    \[ a_k = -cos(\pi k) \]

    \[ -cos(\pi k) = - \frac{1}{2} e^{j \pi k} - \frac{1}{2} e^{-j \pi k} \]

    \[ x(t) = (-\frac{1}{2})  \sum_{k=-\infty}^{\infty}  e^{j k(\omega_0 + \pi) t} + e^{- j k (\omega_0 - \pi) t} \]

    \[\omega_0 = 2\pi / T = \pi / 2 \]

    \[ x(t) = (-\frac{1}{2})  \sum_{k=-\infty}^{\infty}  e^{j k(3\pi / 2) t} + e^{- j k (\pi / 2) t} \]

\item %write the solution of q2  
	\begin{enumerate}
    % Write your solutions in the following items.
    \item %write the solution of q2a
  
    \[ a_k =  \frac{1}{4} \int^4_0 x(t) e^{-jk \omega_o t } dt \]
    \[ a_k =  \frac{1}{4} \int^2_0 2t e^{-j k \omega_o t } dt + \frac{1}{4} \int^4_2 (4-t) e^{-j k \omega_o t }dt \]
    \[ \textbf{By evaluating the integrals, we get:} \]
    \[ a_k = -\dfrac{\left(jkwt+1\right)\mathrm{e}^{-jkwt}}{2j^2k^2w^2}+ \dfrac{\left(jkw\cdot\left(t-4\right)+1\right)\mathrm{e}^{-jkwt}}{4j^2k^2w^2}\]

    
    \item %write the solution of q2b
    \[x(t)  \longleftrightarrow_{FS} a_k\]
    \[\frac{dx(t)}{dt} \longleftrightarrow_{FS} jk\omega_o a_k\]

    \[{a`}_k = jk\omega_o a_k \]
    
    \[ {a`}_k = -\dfrac{\left(jkwt+1\right)\mathrm{e}^{-jkwt}}{2jkw}+ \dfrac{\left(jkw\cdot\left(t-4\right)+1\right)\mathrm{e}^{-jkwt}}{4jkw} \]
    
    \end{enumerate}






\item %write the solution of q3



    \begin{enumerate}
    % Write your solutions in the following items.


    
    
    \item %write the solution of q3a
    \[ a_k = \frac{1}{N}\sum_{n=0}^{N-1}x[n]e^{-jk(2\pi/N)n} \] 
    
    \[\textbf{a-1) Spectral coefficients of } x_1[n] = cos(\frac{\pi}{2}n)  \]
    \[\frac{\pi}{2}N_0 = 2\pi m \rightarrow N_0 = 4\]
    \[a_k = \frac{1}{4}\sum_{n=0}^{3}cos(\frac{\pi}{2}n)e^{-jk(\pi/2)n} \]
    \[a_k = \frac{1}{4}(1 + 0 + (-1)e^{-jk\pi} + 0 )\]
    \[a_k = \frac{1 - (-1)^k}{4} \]


    \[a_k = \frac{1 - (-1)^k}{4} \]

    \[ a_k = \begin{cases} 
        0 & \textbf{k is even} \\
        0.5 & \textbf{k is odd}
        \end{cases}
    \]

    \[\textbf{a-2) Spectral coefficients of } x_2[n] = sin(\frac{\pi}{2}n)  \]
    \[\frac{\pi}{2}N_0 = 2\pi m \rightarrow N_0 = 4\]
    \[a_k = \frac{1}{4}\sum_{n=0}^{3}sin(\frac{\pi}{2}n)e^{-jk(\pi/2)n} \]
    \[a_k = \frac{1}{4}(0 + e^{-jk(\pi/2)} + 0 - e^{-jk(3\pi/2)} )\]
    \[a_k = \frac{ e^{-jk(\pi/2)} - e^{-jk(3\pi/2)} }{4} \]

    \[\textbf{a-3) Spectral coefficients of } x_3[n] = cos(\frac{\pi}{2}n)sin(\frac{\pi}{2}n)  \]

    \[ x_3[n] = 0 \text{   for all integer n.} \]
    
    
    \item %write the solution of q3b
    \[x_1(t)  \longleftrightarrow_{FS} a_k\]
    \[x_2(t)  \longleftrightarrow_{FS} b_k\]
    \[x(t)  \longleftrightarrow_{FS} a_k \ast b_k\]
    \[ a_k \ast b_k = \sum_{l=0}^3 = a_l b_{k-l}\]
    \[ = a_0b_3 + a_1b_2 + a_2b_1 + a_3b_0 \]
    \[ = 0(b_3) + (a_1)0 + 0(b_1) + (a_3)0 \]
    \[ = 0 \]
    \[ \text{The result is the same due to the multiplication property.  } \]
    \[ \text{The signal that is the multiplication of the signals in time domain has the } \]
    \[ \text{coefficients in frequency domain which are the  } \]
    \[ \text{convolution of the coefficients of these two signals.  } \]
    
    \end{enumerate}

\item %write the solution of q4
\[x_1(t) = cos(k\frac{\pi}{3}) \]
\[x_1(t)  \longleftrightarrow_{FS} b_k\]
\[cos(k\frac{\pi}{3}) = \frac{1}{2} e^{jk(\pi/3)} + \frac{1}{2} e^{-jk(\pi/3)}\]
\[cos(k\frac{\pi}{3})  \rightarrow N = 6 \]
\[b_k = \frac{1}{6}\sum_{n=0}^5x_1[n]e^{-jkn(\pi/3)} = \frac{1}{2} e^{jk(\pi/3)} + \frac{1}{2} e^{-jk(\pi/3)}\]
\[x_1[-1] = x_1[1] = 3 \]


\[x_2(t) = cos(k\frac{\pi}{4})\]
\[x_2(t)  \longleftrightarrow_{FS} c_k\]
\[cos(k\frac{\pi}{4}) = \frac{1}{2} e^{jk(\pi/4)} + \frac{1}{2} e^{-jk(\pi/4)}\]
\[cos(k\frac{\pi}{4})  \rightarrow N = 8\]
\[c_k = \frac{1}{8}\sum_{n=0}^7x_2[n]e^{-jkn(\pi/4)} = \frac{1}{2} e^{jk(\pi/4)} + \frac{1}{2} e^{-jk(\pi/4)} \]
\[x_2[-1] = x_2[1] = 4 \] 

\[ \textbf{By linear property:}\]
\[ x[n] = 4\delta[n+1] + 4\delta[n-1] + 3\delta[n+3] + 3\delta[n-3] \]



\item %write the solution of q5  
    \begin{enumerate}   
    % Write your solutions in the following items.
    
    
    
    \item %write the solution of q5a
    \[x[n] = sin(\frac{6\pi}{13}n + \frac{\pi}{2}) = cos(\frac{6\pi}{13}n)\]
    
    \[\frac{6\pi}{13}N_0 = 2\pi m  \rightarrow N_0 = 13 \]
    
    \item %write the solution of q5b
    \[cos(\frac{6\pi}{13}n) = \frac{1}{2} e^{j(6\pi/13)n} + \frac{1}{2} e^{-j(6\pi/13)n}\]

    \[a_{-1} = a_1 = \frac{1}{2} \]
    \[ a_k = 0 \text{ for any other integer k value.} \]

    
\begin{filecontents}{q5.dat}
 n   xn 
 -2  0
 -1  0.5  
 0   0
 1  0.5
 2  0
 3  0
 
\end{filecontents}

\begin{figure} [h!]
    \centering
    \begin{tikzpicture}[scale=1.0] 
      \begin{axis}[
          axis lines=middle,
          xlabel={$n$},
          xtick={-2, -1, 0, 1 , 2},
          ymin=-2, ymax=2,
          xmin=-3, xmax=3,
          every axis x label/.style={at={(ticklabel* cs:1.05)}, anchor=west,},
          every axis y label/.style={at={(ticklabel* cs:1.05)}, anchor=south,},
          grid,
        ]
        \addplot [ycomb, black, thick, mark=*] table [x={n}, y={xn}] {q5.dat};
      \end{axis}
    \end{tikzpicture}
    \label{fig:q3}
\end{figure}
        
    \end{enumerate} 
    
\item %write the solution of q6
    \begin{enumerate}
    % Write your solutions in the following items.
    \item %write the solution of q6a
    \[ H(j\omega) = (\frac{1}{4}) \frac{1}{\frac{3}{4}+j\omega }\]

    \[ \textbf{Using inverse transform (by table 4.2) : }\]
    \[ \frac{1}{(a+j\omega)}  \longleftrightarrow_{FS^{-1}} e^{-at} \mu(t) \]
    
    \[ h(t) = (\frac{1}{4}) e^{ (-3/4) t} \mu(t) \]

    \[ h(t) = \frac{e^{ (-3/4) t} \mu(t)}{4} \]
    
    \item %write the solution of q6b

    \[ y(t) = x(t) \ast h(t) \]
    \[ =\int_{-\infty}^{\infty} x(\tau) e^{( (-3/4)t - \tau )} \mu(t - \tau) d\tau\]
    \[ x(t) = e^{at} \mu(t) \]
    \[ =\int_{-\infty}^{\infty} e^{a\tau} \mu(\tau) e^{( (-3/4)t - \tau )} \mu(t - \tau) d\tau\]
            
    \[ =\int_{0}^{t} e^{a\tau} e^{( (-3/4)t - \tau )} d\tau\]

    \[ e^{-5t} - e^{-10t} = e^{(-3/4)t} \int_{0}^{t} e^{(a-1)\tau} d\tau\]
    
    
    \end{enumerate}
    
\item %write the solution of q7    

\end{enumerate}


\end{document}


